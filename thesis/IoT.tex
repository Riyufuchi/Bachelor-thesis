%!TeX root =  thesis.tex

\section{Úvod do IoT}

\section{Jednodeskové počítače v kontextu IoT}
% Tato sekce by měla obsahovat informace o různých typech jednodeskových počítačů (např. Raspberry Pi, Arduino) a jejich využití v IoT.

\subsection{Raspberry Pi}
% Podsekce popisující vlastnosti a možnosti Raspberry Pi v IoT.

\subsection{Arduino}
% Podsekce popisující vlastnosti a možnosti Arduino v IoT.
\subsection{Arduino vs. Raspberry Pi}

%\section{Technologie v IoT}
% V této části popište různé technologie a protokoly používané v IoT, jako je MQTT, CoAP, Zigbee, BLE atd.

%\subsection{MQTT}
% Podsekce popisující MQTT a jeho využití v IoT.

%\subsection{CoAP}
% Podsekce popisující CoAP a jeho využití v IoT.

%\subsection{Zigbee}
% Podsekce popisující Zigbee a jeho využití v IoT.

%\subsection{BLE (Bluetooth Low Energy)}
% Podsekce popisující BLE a jeho využití v IoT.

\section{Výzvy v IoT}
% Zde popište výzvy spojené s implementací a provozem IoT systémů, jako je zabezpečení, škálovatelnost, interoperabilita atd.