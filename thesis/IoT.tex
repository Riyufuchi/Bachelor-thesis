%!TeX root =  thesis.tex

\section{Úvod do IoT}

\section{Jednodeskové počítače v kontextu IoT}
% Tato sekce by měla obsahovat informace o různých typech jednodeskových počítačů (např. Raspberry Pi, Arduino) a jejich využití v IoT.

\subsection{Raspberry Pi}
% Podsekce popisující vlastnosti a možnosti Raspberry Pi v IoT.

\subsection{Arduino}
Arduino je open source platforma, v základu se jedná o mikrokontrolér se zabudovaným programátorem, tudíž není potřeba externí programátor čipů, což usnadňuje prototypování různých zařízení. Adrudio mikrokontroléry jsou různé, nejpopulárnější je “Arduino Uno” pro tuto bakalářskou práci používám “Arduino Mega 2560”.

\subsection{Arduino vs. Raspberry Pi}
Na závěr porovnání Arduina a Raspberry Pi. Arduino je mikrokontrolér, vykonává instrukce “od zapnutí do vypnutí” a v pozadí není žádný operační systém jako u Raspberry Pi, proto je spolehlivější.

%\section{Technologie v IoT}
% V této části popište různé technologie a protokoly používané v IoT, jako je MQTT, CoAP, Zigbee, BLE atd.

%\subsection{MQTT}
% Podsekce popisující MQTT a jeho využití v IoT.

%\subsection{CoAP}
% Podsekce popisující CoAP a jeho využití v IoT.

%\subsection{Zigbee}
% Podsekce popisující Zigbee a jeho využití v IoT.

%\subsection{BLE (Bluetooth Low Energy)}
% Podsekce popisující BLE a jeho využití v IoT.

\section{Výzvy v IoT}
% Zde popište výzvy spojené s implementací a provozem IoT systémů, jako je zabezpečení, škálovatelnost, interoperabilita atd.