%!TeX root =  thesis.tex

\section{Úvod do IoT}

\section{Jednodeskové počítače v kontextu IoT}
% Tato sekce by měla obsahovat informace o různých typech jednodeskových počítačů (např. Raspberry Pi, Arduino) a jejich využití v IoT.

\subsection{Raspberry Pi}

\begin{figure}[h!]
	\centering
	\includegraphics[width=\textwidth]{pictures/placeHolderFHD.png}
    	\caption{Raspberry Pi}
   	\label{fig:rasPI}
\end{figure}

\subsection{Arduino}
Arduino je open source platforma, v základu se jedná o mikrokontrolér se zabudovaným programátorem, tudíž není potřeba externí programátor čipů, což usnadňuje prototypování různých zařízení. Adrudio mikrokontroléry jsou různé, nejpopulárnější je “Arduino Uno” pro tuto bakalářskou práci používám “Arduino Mega 2560”.

\begin{figure}[h!]
	\centering
	\includegraphics[width=\textwidth]{pictures/placeHolderFHD.png}
    	\caption{Arduino Uno}
   	\label{fig:arduinoUno}
\end{figure}	

\subsection{Arduino vs. Raspberry Pi}
Na závěr porovnání Arduina a Raspberry Pi. Arduino je mikrokontrolér, vykonává instrukce “od zapnutí do vypnutí” a v pozadí není žádný operační systém jako u Raspberry Pi, proto je spolehlivější.


\section{Výzvy v IoT}
\subsection{Hardwarové}
\subsection{Softwarové}
