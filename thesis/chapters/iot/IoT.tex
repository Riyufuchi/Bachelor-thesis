%!TeX root =  ../../thesis.tex

\section{Jednodeskové počítače v kontextu IoT}
Internet věcí (IoT) označuje síť vzájemně propojených fyzických zařízení, která jsou schopna sbírat a sdílet data pomocí vestavěných senzorů a komunikačních technologií. Tyto zařízení mohou být od všedních předmětů až po složité průmyslové systémy.
Základem těchto zařízení bývají jednodeskové počítače nebo po případě mikrokontroléry.\\
Jednodeskové počítače jsou zařízení obsahující veškeré potřebné komponenty pro provoz a řízení výpočetních procesů na jediné desce plošného spoje a mezi známé příklady patří Arduino, Raspberry Pi nebo ESP32.\\


\subsection{Raspberry Pi}

\begin{figure}[h!]
	\centering
	\includegraphics[width=\textwidth]{pictures/placeHolderFHD.png}
    	\caption{Raspberry Pi}
   	\label{fig:rasPI}
\end{figure}

\subsection{Arduino}
Arduino je open source platforma, v základu se jedná o mikrokontrolér se zabudovaným programátorem, tudíž není potřeba externí programátor čipů, což usnadňuje prototypování různých zařízení. Adrudio mikrokontroléry jsou různé, nejpopulárnější je “Arduino Uno” pro tuto bakalářskou práci používám “Arduino Mega 2560”.

\begin{figure}[h!]
	\centering
	\includegraphics[width=\textwidth]{pictures/placeHolderFHD.png}
    	\caption{Arduino Uno}
   	\label{fig:arduinoUno}
\end{figure}	

\subsection{Arduino vs. Raspberry Pi}
Zatímco Raspberry Pi je jednodeskový počítač v tradičním slova smyslu, neboli má grafický výstup zpracovávaný grafickým čipem a operační systém (Linux nebo Windows) a dá se použít jako stan
Na závěr porovnání Arduina a Raspberry Pi. Arduino je mikrokontrolér, vykonává instrukce “od zapnutí do vypnutí” a v pozadí není žádný operační systém jako u Raspberry Pi, proto je spolehlivější.


\section{Výzvy v IoT}
\subsection{Hardwarové}
\subsection{Softwarové}
