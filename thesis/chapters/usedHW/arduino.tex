%!TeX root =  ../../thesis.tex

\section{Arduino}
Základem testeru je mikrokontrolér od značky Arduino model \ardMeg. Tento model disponuje velkým počtem pinům, na které je možné připojit různá zařízení a následně je ovládat mikrokontrolérem. Mikrokontroléry Arduino se programují pomocí C/C++. Já firmware pro toto tester píši v C++.

\begin{figure}[h!]
	\centering
	\includegraphics[width=\textwidth]{pictures/placeHolderFHD.png}
    	\caption{\ardMeg}
   	\label{fig:arduinoMega}
\end{figure}


%\subsection*{Ino file}
%\lstinputlisting[language=C++, caption={Code.ino}, label={lst:code}]{../code/Code.ino}

%\subsubsection*{Co je to Ino file?}
%Ino file je soubor skatche pro Arduino, Arduino IDE ho využívá jako hlavní soubor, jen má místo metody main() metody setup() a loop().  Metoda setup() slouží pro přípravu zařízení a je automaticky volána při spuštění zaříze,  loop() pak obsahuje kód, který je vykonáván %mikrokontrolérem, dokud není vypnut nebo se nevyskytne problém. Arduino se dá programovat i pomocí svého upraveného jazyka C++, proto koncovka .ino.



