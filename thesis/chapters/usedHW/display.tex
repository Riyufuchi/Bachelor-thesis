%!TeX root =  ../../thesis.tex

\section{Display}

Pro potřeby testeru byl vybrán standardní LCD displej se žlutým podsvícením o čtyřech řádcích po dvaceti znacích. Tento displej bude postačovat, jelikož je potřeba zobrazit menu, aby uživatel mohl vybrat typ kabelu dle koncových konektorů a následně výsledek testu, kterému jsou věnovány poslední dva řádky, na které se taky budou vypisovat chybová hlášení. V prvním řádku je vypsána verze programu a na poslední pozici bliká znak “\#”, který se objeví a zmizí vždy na jednu sekundu, z důvodu kontroly, že zařízení “nezamrzlo” a je možné jej bez problému použít.

\begin{figure}[h!]
	\centering
	\includegraphics[width=\textwidth]{pictures/placeHolderFHD.png}
    	\caption{Display HW}
   	\label{fig:displayHW}
\end{figure}