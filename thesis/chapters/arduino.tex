%!TeX root =  ../thesis.tex

\section{Casing}

\section{Arduino}
Platforma Arduino se programuje pomocí C/C++ popřípadě Arduino verze C/C++. Já firmaware pro toto zařízení napíši v C++, ačkoliv by se většinou běžně vyvíjí v Céčku, ale pro tento projekt poslouží C++ lépe a taky s ním mám více zkušeností.

\subsection{Ino file}
\lstinputlisting[language=C++, caption={Code.ino}, label={lst:code}]{../code/Code.ino}

\subsubsection{Co je to Ino file?}
Ino file je soubor skatche pro Arduino, Arduino IDE ho využívá jako hlavní soubor, jen má místo metody main() metody setup() a loop().  Metoda setup() slouží pro přípravu zařízení a je automaticky volána při spuštění zaříze,  loop() pak obsahuje kód, který je vykonáván mikrokontrolérem, dokud není vypnut nebo se nevyskytne problém. Arduino se dá programovat i pomocí svého upraveného jazyka C++, proto koncovka .ino.

\newpage
\subsection{Controller}
\lstinputlisting[language=C++, caption={Device controller}, label={lst:cpp_controller}]{../code/Controller.h}
Controller je hlavní třída, ačkoliv je hlavní část je v code.ino \ref{lst:code}, třída Controller zpřehlední kód, jelikož se do ní nic nedoplňuje při kompilaci, jako do hlavního ino souboru.