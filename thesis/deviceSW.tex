%!TeX root =  thesis.tex
\section{Jazyk a struktura}
Platforma Arduino se programuje pomocí C/C++ popřípadě Arduino verze C/C++. Já firmaware pro toto zařízení napíši v C++, ačkoliv by se většinou běžně vyvíjí v Céčku, ale pro tento projekt poslouží C++ lépe a taky s ním mám více zkušeností.
\newpage
\subsection{Controller}
\lstinputlisting[language=C++, caption={Device controller}, label={lst:cpp_controller}]{../code/Controller.h}

\section{Interakce s uživatelem}
Uživatel ovládá zařízení skrze jednoduchou klávesničku a výstpu je na LCD displeji.

\newpage
\section{Display}
\subsection{Display controller}
\lstinputlisting[language=C++, caption={Display controller}, label={lst:cpp_display}]{../code/Display.h}

\newpage
\section{Klávesnička}
\subsection{Keyboard controller}
\lstinputlisting[language=C++, caption={Display controller}, label={lst:cpp_display}]{../code/Keyboard.h}

\subsection{Připojení k desce}
\begin{table} [h!]
	\centering
	\catcode`\-=12 % Because of czech
	\begin{tabular}[c]{|| c | c | c ||}
	\hline
		\multicolumn{3}{||c||}{Piny klávesničky} \\
	\hline
 		 \textbf{PIN} & \textbf{Tlačítko} & \textbf{Akce}\\
	\hline
		42 &  RIGHT & Posune menu doprava\\
	\hline
		46 & POWER & N/A\\
	\hline
		48 & LEFT & Posune menu doleva\\
	\hline
		50 & MENU & N/A\\
	\hline
		52 & AUTO & Začne testovat vybraný kabel\\
	\hline
	\end{tabular}
	\caption{Pinové rozložení klávesnice}
	\label{table:pinKB}
\end{table}

\section{Testování kabelů}
\newpage
\subsection{Předek pro třídy konnectorů}
\lstinputlisting[language=C++, caption={IConnector}, label={lst:cpp_IConnector}]{../code/IConnector.h}
Toto je společný předek pro jednotlivé konektory, který nám usnadní manipulaci s jeho potomky, tedy již specifickými konektory.
