%!TeX root =  thesis.tex

\section{Návrh softwarové části}

\subsection{Jazyk a struktura}
Platforma Arduino se programuje pomocí C/C++ popřípadě Arduino verze C/C++. Já firmaware pro toto zařízení napíši v C++, ačkoliv by se většinou běžně vyvíjí v Céčku, ale pro tento projekt poslouží C++ lépe a taky s ním mám více zkušeností.
\newpage
\subsubsection{Controller}
\lstinputlisting[language=C++, caption={Device controller}, label={lst:cpp_controller}]{../code/Controller.h}

\subsection{Interakce s uživatelem}
Uživatel ovládá zařízení skrze jednoduchou klávesničku a výstpu je na LCD displeji.

\newpage
\subsubsection{Display controller}
\lstinputlisting[language=C++, caption={Display controller}, label={lst:cpp_display}]{../code/Display.h}



\subsection{Testování kabelů}
\newpage
\subsubsection{Předek pro třídy konnectorů}
\lstinputlisting[language=C++, caption={IConnector}, label={lst:cpp_IConnector}]{../code/IConnector.h}
Toto je společný předek pro jednotlivé konektory, který nám usnadní manipulaci s jeho potomky, tedy již specifickými konektory.
