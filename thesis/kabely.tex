%!TeX root =  thesis.tex

\section{IO Panel}
\begin{figure}[h!]
	\centering
	\includegraphics[width=\textwidth]{pictures/placeHolderFHD.png}
    	\caption{IO Panel}
   	\label{fig:panelIO}
\end{figure}

\section{IConnector}
\lstinputlisting[language=C++, caption={IConnector}, label={lst:cpp_IConnector}]{../code/src/deviceIO/IConnector.h}
Toto je společný předek pro jednotlivé konektory, který nám usnadní manipulaci s jeho potomky, tedy již specifickými konektory.

\section{XLR 3}
\subsection{Pinové zapojení}
\begin{table} [h!]
	\centering
	\catcode`\-=12 % Because of czech
	\begin{tabular}[c]{|| c | c ||}
	\hline
		\multicolumn{2}{||c||}{XLR-3 In} \\
	\hline
 		 \textbf{PIN} & \textbf{Akce}\\
	\hline
		22 &  Uzemění\\
	\hline
		24 & Positive polarity terminal for balanced audio circuits \\
	\hline
		N/A & Negative polarity terminal for balanced circuits \\
	\hline
	\end{tabular}
	\caption{Pinové rozložení XLR 3 - In}
	\label{table:pinXLR-IN}
\end{table}

\begin{table} [h!]
	\centering
	\catcode`\-=12 % Because of czech
	\begin{tabular}[c]{|| c | c ||}
	\hline
		\multicolumn{2}{||c||}{XLR-3 Out} \\
	\hline
 		 \textbf{PIN} & \textbf{Akce}\\
	\hline
		26 &  Uzemění\\
	\hline
		28 & Positive polarity terminal for balanced audio circuits \\
	\hline
		N/A & Negative polarity terminal for balanced circuits \\
	\hline
	\end{tabular}
	\caption{Pinové rozložení XLR 3 - Out}
	\label{table:pinXLR-OUT}
\end{table}

\section{Jack 2.1}
\subsection{Pinové zapojení}

\section{Jack 2.5}
\subsection{Pinové zapojení}

\section{RCA}
\subsection{Pinové zapojení}

\section{BOSH}
\subsection{Pinové zapojení}

\section{Shimano}
\subsection{Pinové zapojení}
