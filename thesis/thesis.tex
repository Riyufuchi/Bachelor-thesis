\documentclass[12pt,a4paper,titlepage]{scrreprt}

% Language is defined in package module

\usepackage {fontspec}
\usepackage[hidelinks]{hyperref}
\usepackage{pdfpages} % For inclusion of PDFs

\usepackage{tabulary}
\usepackage{emoji}

%pro formátování čísel
\usepackage{numprint}
\npthousandsep{\,}\npthousandthpartsep{}\npdecimalsign{.}

%Uprava (odstraneni modifikaci generovaneneho texttu tridou scratcl)
\setkomafont{disposition}{\mdseries\rmfamily}

\title{\vspace{6cm}Návrh zařízení pro testování nabíjecích kabelů}
\subtitle{Bakalářská práce}
\author{Filip Šimek}
\date{\today}

\hypersetup{
	pdfauthor={Filip Šimek},%
	pdftitle={Návrh zařízení pro testování nabíjecích kabelů},%
	pdfsubject={Bakalářská práce},%
	%pdfkeywords={one, two},%
	%pdfproducer={LaTeX},%
	pdfcreator={LuaLaTeX}
}

%font
\setmainfont{Times_New_Roman}

%Rozestup řádku (1.5)
%\linespread{1.35}
%\linespread{1.5}
\linespread{1.25}

%Nastaveni okraje (normostrana)
\usepackage[a4paper, left=3.50cm, right=1.50cm, top=2.50cm, bottom=2.50cm]{geometry}

\RedeclareSectionCommand[
  beforeskip=.1cm,
  afterskip= .1cm %1.0ex plus .2ex
]{chapter}

%\setkomafont{chapter}{\fontsize{18}{18*1.2}\selectfont}
\setkomafont{chapter}{\fontsize{16}{19.2}\selectfont}   
\setkomafont{section}{\fontsize{14}{16.8}\selectfont}
\setkomafont{subsection}{\fontsize{13}{15.6}\selectfont}
\setkomafont{subsubsection}{\fontsize{13}{15.6}\selectfont}

\makeatletter
\newcommand\thefontsize{(Font size = \f@size pt)}
\makeatother

% WIP modules
%\usepackage{glossaries}
%%!TeX root =  ../thesis.tex

\makeglossaries

\newglossaryentry{keyword1}{
    name={Keyword 1},
    description={Description of Keyword 1}
}

\newglossaryentry{keyword2}{
    name={Keyword 2},
    description={Description of Keyword 2}
}

\newglossaryentry{ard}{
    name={Arduino},
    description={IoT device}
}
%\makeglossaries

% My modules
% Přidá automatické zalomení spojek

\usepackage[czech]{babel}

% Define custom hyphenation rules
\babelhyphenation[czech]{a}
\babelhyphenation[czech]{u}
\babelhyphenation[czech]{i}
\babelhyphenation[czech]{v}
\babelhyphenation[czech]{s}
\input{texModules/paragraph.tex}
\input{texModules/cppListing.tex}
\input{texModules/table.tex}
\input{texModules/justify.tex}

% My commands
\newcommand{\ardMeg}{Arduino Mega 2560}

\begin{document}

	%\includepdf{specialPages/frontPage/frontPage.pdf} %Desky
	\includepdf{specialPages/titlePage/titlePage.pdf} %\maketitle 
	\includepdf[pages={1-2}]{specialPages/zadaniBP.pdf}
	
	%!TeX root =  ../thesis.tex

Prohlašuji:
\\
Práci s názvem “Návrh zařízení pro testování nabíjecích kabelů” jsem vypracoval samostatně. Veškeré literární prameny a informace, které jsem v práci využil, jsou uvedeny v seznamu použité literatury.
\\
Byl jsem seznámen s tím, že se na moji práci vztahují práva a povinnosti vyplývající ze zákona č. 121/2000 Sb., o právu autorském, o právech souvisejících s právem autorským a o změně některých zákonů (autorský zákon), ve znění pozdějších předpisů, zejména se skutečností, že Univerzita Pardubice má právo na uzavření licenční smlouvy o užití této práce jako školního díla podle § 60 odst. 1 autorského zákona, a s tím, že pokud dojde k užití této práce mnou nebo bude poskytnuta licence o užití jinému subjektu, je Univerzita Pardubice oprávněna ode mne požadovat přiměřený příspěvek na úhradu nákladů, které na vytvoření díla vynaložila, a to podle okolností až do jejich skutečné výše.
\\
Beru na vědomí, že v souladu s § 47b zákona č. 111/1998 Sb., o vysokých školách a o změně a doplnění dalších zákonů (zákon o vysokých školách), ve znění pozdějších předpisů, a směrnicí Univerzity Pardubice č. 7/2019 Pravidla pro odevzdávání, zveřejňování a formální úpravu závěrečných prací, ve znění pozdějších dodatků, bude práce zveřejněna prostřednictvím Digitální knihovny Univerzity Pardubice.\\\\
V Pardubicích dne \hfill Filip Šimek v. r. 2024
	\thispagestyle{empty}
	%!TeX root =  ../thesis.tex
\clearpage
\vspace*{\fill}
\section*{Poděkování}
Mé poděkování patří vedoucímu bakalářské práce Ing. Janu Fikejzovi, Ph.D. za odborné vedení, trpělivost, ochotu a veškerý věnovaný čas, který mi v průběhu zpracování bakalářské práce věnoval. Dále bych chtěl poděkovat rodině a svým kamarádům, kteří mi byli po celou dobu studia oporou.
	\thispagestyle{empty}
	
	\setcounter{page}{5}
	\newpage
	%!TeX root =  ../thesis.tex

\section*{List of Shortcuts}

\begin{description}
    \item[Ctrl+C] Copy selected text
    \item[Ctrl+X] Cut selected text
    \item[Ctrl+V] Paste copied or cut text
    \item[Ctrl+S] Save current file
    \item[Ctrl+Z] Undo last action
    \item[Ctrl+Y] Redo last undone action
\end{description} %\printglossary
	\thispagestyle{empty}
	
	
	\addtocontents{toc}{\protect\pagestyle{empty}}
	\addtocontents{toc}{\protect\thispagestyle{empty}}
	\tableofcontents
	\thispagestyle{empty}
	
	\listoffigures
	\addcontentsline{toc}{section}{Seznam obrázků}
	\thispagestyle{empty}
	\listoftables
	\addcontentsline{toc}{section}{Seznam tabulek}
	\thispagestyle{empty}
	
	\chapter{Úvod} %\thefontsize}
	V posledních letech se rozšířila elektromobilita, zejména elektro-koloběžky a elektrická kola, která jsou ve spoustě městech i sdílená. Tyto dopravní prostředky je potřeba nabíjet, ovšem při výrobě nabíjecích kabelů může dojít k chybě a proto je potřeba je otestovat, aby se zaručila jejich správná funkčnost. Špatný kabel nemusí fungovat vůbec nebo při špatném zapojení koncového konektoru může způsobit poškození daného zařízení. Tato bakalářská práce má za cíl v praktické části navrhnout a sestavit zařízení na testování kabelů, které bude použito při výrobě kabelů k ověření jejich funkčnosti. V teoretické části se práce zabývá IoT, nebo-li Internet věcí (Internet of Things) a technologiemi, které budou použity pro sestavení testovacího zařízení.
	
	\chapter{IoT - Internet věcí}
	%!TeX root =  thesis.tex

\section{Úvod do IoT}

\section{Jednodeskové počítače v kontextu IoT}
% Tato sekce by měla obsahovat informace o různých typech jednodeskových počítačů (např. Raspberry Pi, Arduino) a jejich využití v IoT.

\subsection{Raspberry Pi}
% Podsekce popisující vlastnosti a možnosti Raspberry Pi v IoT.

\subsection{Arduino}
% Podsekce popisující vlastnosti a možnosti Arduino v IoT.
\subsection{Arduino vs. Raspberry Pi}

%\section{Technologie v IoT}
% V této části popište různé technologie a protokoly používané v IoT, jako je MQTT, CoAP, Zigbee, BLE atd.

%\subsection{MQTT}
% Podsekce popisující MQTT a jeho využití v IoT.

%\subsection{CoAP}
% Podsekce popisující CoAP a jeho využití v IoT.

%\subsection{Zigbee}
% Podsekce popisující Zigbee a jeho využití v IoT.

%\subsection{BLE (Bluetooth Low Energy)}
% Podsekce popisující BLE a jeho využití v IoT.

\section{Výzvy v IoT}
% Zde popište výzvy spojené s implementací a provozem IoT systémů, jako je zabezpečení, škálovatelnost, interoperabilita atd.
	%!TeX root =  ../../thesis.tex

\section{Výzvy v IoT}
Vývoj zařízení v rámci Internetů věcí sebou přináší spoustu výzev, jak v rámci designu hardwaru, tak správného a efektivního využití limitovaných prostředků při návrhu softwaru. 

\subsection{Hardwarové}
Hardwarové výzvy jsou především ovlivněny potřebnou velikostí, kdy se na malý deskový počítač musí všechny komponenty a IO pro komunikaci s dalšími zařízeními jako jsou například senzory a ovladače na motorky. Na návrhu hardwaru IoT zařízení je nejtišší navrhnout “tělo” zařízení, kdy například pro dron je důležitá výdrž baterky, ale i jeho samotná váha, což znamená použít lehké, však dostatečně pevný materiál. Je potřeba brát v potaz mnoho atributů zařízení, kdy navýšením jednoho můžeme snížit druhý, například větší baterka znamená delší výdrž, ale také větší váhu.

\subsection{Softwarové}
Při návrhu softwaru pro mikrokontroléry je potřeba dobře zvážit datové typy, jednak aby nedocházelo k přetečení a pak aby se neplýtvalo pamětí, které je podstatně výrazně méně než na moderních počítačích. Pro maximální efektivitu se člověk nevyhne nutné znalosti programovacího jazyka C, jelikož právě skrze C máme největší kontrolu nad hardwarovými prostředky (Assembler neberu v potaz, jelikož se dnes běžně nepoužívá). Při návrhu softwaru pro mikrokontroléry je nejlepší se vyhnout dynamickému alokování paměti, jelikož by nám tímto způsobem mohla rychle dojít, nejlepší je použít buffery, hlavně při práci s textovými řetězci. S těmi se pracuje v takzvaném C-stylu, neboli jako s polem znaků, jelikož C++ textový řetězec std::string je objekt a zabere podstatně více paměti než samotné pole znaků. Sám sem se potýkal s problémy při vývoji softwaru pro tester, kdy mi přetékal buffer na textový řetězec nebo nastal zmatek při práci s odkazem na buffer a ten byl přemazán a výpis na displej nebyl správný.

\subsubsection{Debuggování}
Debugovat mikrokontrolér není tak lehké jako program pro počítač, ale není to nemožné a má-li zařízení možnost komunikace s počítačem, jako například v případě Arduina, můžeme využít sériovou komunikaci skrze USB kabel, kterým do něj nahráváme kód. Jinak se dá využít klasické debuggování pomocí takzvaného krokování.
\begin{figure}[H]
	\centering
	\includegraphics[width=0.9\textwidth]{pictures/code.png}
    	\caption{Ukázka kódu kdy Arduino posílá zprávu do IDE}
   	\label{fig:usbIDE}
\end{figure}

Na obrázku \ref{fig:usbIDE} je kód, který demonstruje jak se dá poslat text do IDE. V setupu je potřeba nastavit modulační rychlost, kdy 9600 je maximum pro Arduino. V metodě \hl{loop()} je pak samotný výstup, kdy sekundové čekání je moc, ale měl by se text vypisovat opakovaně, tak bude potřeba, jinak ho nebudeme stíhat číst.

\begin{figure}[H]
	\centering
	\includegraphics[width=0.9\textwidth]{pictures/message.png}
    	\caption{Ukázka výpisu zprávy z Arduina}
   	\label{fig:msgIDE}
\end{figure}
Na obrázku \ref{fig:msgIDE} vidíme konzoli, která je ve spodní části IDE a je v ní text “Hello world!” z ukázky na obrázku \ref{fig:usbIDE}.


	
	
	\chapter{Použitý HW a SW}
	%!TeX root =  thesis.tex

\section{Casing}
\section{Arduino}
Platforma Arduino se programuje pomocí C/C++ popřípadě Arduino verze C/C++. Já firmaware pro toto zařízení napíši v C++, ačkoliv by se většinou běžně vyvíjí v Céčku, ale pro tento projekt poslouží C++ lépe a taky s ním mám více zkušeností.
\subsection{Controller}
\lstinputlisting[language=C++, caption={Device controller}, label={lst:cpp_controller}]{../code/Controller.h}
\newpage
Controller je hlavní třída, ačkoliv je hlavní část je v code.ino \ref{lst:code}, třída Controller zpřehlední kód, jelikož se do ní nic nedoplňuje při kompilaci. Kód v code.ino obsahuje metody setup() a loop(). Metoda setup() slouží pro přípravu zažízení a je automaticky volána při spuštění zařízeni. Metoda loop() pak obsahuje kód, který je vykonáván mikrokontrolérem, dokud není vypnut nebo se nevyskytne problém. Arduino se dá programovat i pomocí jejich upraveného jazyka Cé, proto koncovka .ino. 
\lstinputlisting[language=C++, caption={Code.ino}, label={lst:code}]{../code/Code.ino}
	\newpage
	%!TeX root =  ../../thesis.tex

\section{Display}

Pro potřeby testeru byl vybrán standardní LCD displej se žlutým podsvícením o čtyřech řádcích po dvaceti znacích. Tento displej bude postačovat, jelikož je potřeba zobrazit menu, aby uživatel mohl vybrat typ kabelu dle koncových konektorů a následně výsledek testu, kterému jsou věnovány poslední dva řádky, na které se taky budou vypisovat chybová hlášení. V prvním řádku je vypsána verze programu a na poslední pozici bliká znak “\#”, který se objeví a zmizí vždy na jednu sekundu, z důvodu kontroly, že zařízení “nezamrzlo” a je možné jej bez problému použít.

\begin{figure}[H]
	\centering
	\includegraphics[width=0.9\textwidth]{pictures/display.jpeg}
    	\caption{Display}
   	\label{fig:displayHW}
\end{figure}

Na obrázku \ref{fig:displayHW} je displej, který sice není použitý pro tester, ale je to principiálně stejný displej a vlevo nahoře jsou vidět vrchní strany pinů, kterými se displej připojuje k Arduino nebo po případě k breadbordu.
	\newpage
	%!TeX root =  ../../thesis.tex

\section{Klávesnice}
Pro dostatečné ovládání byla vybrána klávesnice s pěti tlačítky: Right, Left, Power, Auto, Menu. Tlačítko “Power” začne test kabelu, tlačítka “Left” a “Right” se používají k posouvání rotačního meny a tlačítko “Menu” přepíná mezi třemi zdrojovými konektory, nebo-li přepíná mezi submeny.

\begin{figure}[H]
	\centering
	\includegraphics[width=0.9\textwidth]{pictures/keyboard.jpeg}
    	\caption{Klávesnice testeru}
   	\label{fig:keyborad}
\end{figure}

Na obrázku \ref{fig:keyborad} je vidět klávesnice/ovládací panel, kterým se tester ovládá.

\subsection*{Připojení k desce}
\begin{table} [h!]
	\caption{Pinové rozložení klávesnice}
	\label{table:pinKB}
	\centering
	\catcode`\-=12 % Because of czech
	\begin{tabular}[c]{|| c | c | c | c ||}
	\hline
		\multicolumn{4}{||c||}{Piny klávesničky} \\
	\hline
		\textbf{PIN} & \textbf{Tlačítko} & \textbf{Barva kabelu} & \textbf{Akce}\\
	\hline
		42 &  RIGHT & \textcolor{purple}{fialová} & Posune rotační menu doprava\\
	\hline
		46 & POWER & \textcolor{green}{zelená} & Začne testovat vybraný kabel\\
	\hline
		48 & LEFT & \textcolor{yellow}{\textbf{žlutá}} & Posune rotační menu doleva\\
	\hline
		50 & MENU & \textcolor{orange}{oranžová} & Přepne menu na další zdrojový konektor \\
	\hline
		52 & AUTO & \textcolor{red}{červená} & Začne testovat vybraný kabel\\
	\hline
		44 & - & \textcolor{blue}{modrá} & Napájení klávesnice\\
	\hline
		GND & - & \textcolor{gray}{šedá} & Uzemnění\\
	\hline
	\end{tabular}
\end{table}

V tabulce  \ref{table:pinKB} je popsáno, které piny jsou napojeny na konkrétní tlačítko a co je akce vykonávaná při stisku tlačítka.

	
	\chapter{Návrh testeru kabelů – Hardwarová část}
	%!TeX root =  ../../thesis.tex

\section{Tester}
Tester je umístěn v plastovém boxu na vrchní části je umístěn displej a klávesnice na ovládání a na přední části jsou konektory pro testovaní kabelů.
\begin{figure}[h!]
	\centering
	\includegraphics[width=\textwidth]{pictures/tester-top.jpeg}
    	\caption{Výsledný tester na kabely}
   	\label{fig:tester}
\end{figure}


	
	\chapter{Návrh testeru kabelů – Softwarová část}
	%!TeX root =  ../../thesis.tex

\section{Klíčové třídy}
\subsection{Controller}
Controller, jak název napovídá, je hlavní třída, která řídí běh zařízení a ačkoliv hlavní část je v code.ino, třída Controller zpřehlední kód, jelikož podléhá standardnímu zápisu C++ kódu. Třída Controller obsahuje dvě důležité veřejné metody a to metodu \hl{run()} a \hl{initialize()}.

\subsubsection{Metoda initilize()}
Tato metoda inicializuje zařízení a volá metody pro inicializaci na komponentech, v případě, že nastane chyba, zařízení se jí pokusí ohlásit, výpisem na displej a zvukovým signálem, v intervalu SOS v morseovce. Tato metoda je volána v metodě \hl{setup()} při přípravě zařízení a na jejím konci se při úspěchu ozve bzučák s krátkým pípnutím, které oznámí, že zařízení je připravené k použití.

\subsubsection{Metoda run()}
Tato metoda hlavní činnost programu pro tester a je volána v metodě \hl{loop()}. Na začátku metody se zkontroluje vstup z klávesnice, pokud nějaký je, tak se vykoná příslušná akce, pak se řeší probliknutí kontrolního znaku na displeji.

\subsection{Connector}
Třída Connector slouží jako předek pro konkrétní konektory, ačkoliv není abstraktní a dá se vytvořit instance, tak dědění nám zpřehledňuje kód, už jen kvůli tomu, že vždy nemusíme vyplnit celý její konstruktor, který si žádá mnoho parametrů, jako jsou například název, pole s čísly příslušných pinů, odkaz na proti konektor a atp.. Tato třída obsahuje také společné metody pro práci s konektory, které si nyní popíšeme.

\subsubsection{void setMode(Mode mode)}
Na obrázku \ref{fig:setMode} je implementace metody, která nastavuje piny, na které je konektor připojen, na vstupní nebo výstupní.  Rozhoduje pomocí hodnoty parametru, kterým je enumerátor Mode. Použití enumerátoru dělá kód přehlednější a dá se s ním lépe pracovat oproti aliasům na režimy definované v \hl{Arduino.h}. Enumerátor Mode je také atributem třídy Connector a tato metoda slouží zároveň jako setter pro tento atribut.

\begin{figure}[H]
	\centering
	\includegraphics[width=0.9\textwidth]{pictures/setMode.png}
    	\caption{Metoda setMode(Mode)}
   	\label{fig:setMode}
\end{figure}

Metoda je ošetřena proti chybám, ze jména pak proti nulovému ukazateli na pole s čísly pinů. Ačkoliv by se dalo pracovat pouze s číselnou hodnotou aliasů, jak je patrné z implementace, tak enumerátor, jak již bylo zmíněno, dělá kód přehlednější, jelikož například \hl{bosh.setMode(Mode:IN);} je samovysvětlující oproti \hl{bosh.setMode(2);}.

\subsubsection{void reconnectTo(Connector* connector)}
Metoda reconnectTo(Connector*) nastavuje ukazatel na protější konektor a vytváří název spojení, který se bude zobrazovat v menu.

\begin{figure}[H]
	\centering
	\includegraphics[width=0.9\textwidth]{pictures/reconnect.png}
    	\caption{Metoda reconnectTo(Connector*)}
   	\label{fig:reccon}
\end{figure}

Implementace metody na obrázku \ref{fig:reccon} ukazuje, že neslouží jako setter na ukazatel na instanci objektu protějšího konektoru, ale také nastaví/přenastaví jméno spojení, aby bylo bylo možné jej zobrazit v menu, tak, že výstupní konektor je vpravo, tudíž výsledný formát je například \hl{Xlr3~\texttt{->}~Jack 2.1}.

\subsubsection{bool startTest(char resultArr[])}
Toto je metoda na spuštění testu, jako parametr bere pole znaků, ačkoliv se nejedná o textový řetězec, ale o pole s malými čísly. Na první místo se píše stav testu, hodnota 0 je pro úspěšný test a záporné jsou pro chyby. Zbytek pole pak obsahuje buď -100 nebo číslo pinu na konektoru, který je vadný. V případě úspěšného testu je návratová hodnota \textbf{true}, jinak vždy \textbf{false}. Tato metoda také kontroluje, zda-li je test proveditelný. Mezi kontrolované případy patří: nulový ukazatel na protější konektor, špatně nastavený konektor nebo že oba jsou výstupní nebo vstupní. Metodu, která provádí samotné testování popíši v kapitole Testování.

\subsection{IComponent}
Společný předek pro komponenty zařízení, jako jsou například: displej, klávesnice nebo bzučák. Tento předek obsahuje jednu čistě virtuální funkci a to funkci \hl{inicialize()}, která vrací \textbf{true}, když se podaří danou komponentu inicializovat.

	
	
	\chapter{Testování}
	%!TeX root =  ../../thesis.tex

\section{Princip}

	
	
	\chapter{Závěr}
	Cílem práce bylo vytvořit zařízení pro testování nabíjecích kabelů pro elektrická kola s použitím mikrokontroléru Arduino. Výsledné zařízení je splňuje požadavky a je možné jednoduše rozšířit funkčnost díky objektovém přístupu v jazyce C++. Zařízení je schopné vypsat na displej přesně, které piny konektoru jsou špatné a nepřenáší se skrz ně elektrický proud/signál, také vypisuje chybové hlášky i v případě chybné implementace, například při nenastavení příslušných pinů pro konektor na Arduinu. Také pomocí zvuku upozorňuje zařízení uživatele na na výsledek testu a svého stavu, jsou naimplementované/ošetřené čtyři chybové stavy. Také na bezpečnost byl kladen důraz a přes kabel nejde více než 5V, které stačí, pro vyhodnocení testu.

	\newpage
	\chapter{Přílohy}
	\section*{Zdroje}
	\addcontentsline{toc}{section}{Seznam zdrojů}
	https://www.arduino.cc/en/about
	
	%\lstlistoflistings
	%\addcontentsline{toc}{section}{Seznam zdrojových kódů} % Add the list of listings to the TOC as a section

\end{document}