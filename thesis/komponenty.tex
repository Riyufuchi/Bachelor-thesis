%!TeX root =  thesis.tex

\section{IComponent}
\lstinputlisting[language=C++, caption={IComponent}, label={lst:cpp_icomp}]{../code/src/components/IComponent.h}

\newpage
\section{Display}
\begin{figure}[h!]
	\centering
	\includegraphics[width=\textwidth]{pictures/placeHolderFHD.png}
    	\caption{Display HW}
   	\label{fig:displayHW}
\end{figure}

\newpage
\subsection{Display controller}
\lstinputlisting[language=C++, caption={Display controller}, label={lst:cpp_display}, style=riyuCpp]{../code/src/components/Display.h}
Pro ovládáni displaye používám modul, se kterým se kominikuje skrze “LiquidCrystal\_IC2” knihovnu, jejímž autorem je AUTOR.

\section{Klávesnice}
\begin{figure}[h!]
	\centering
	\includegraphics[width=\textwidth]{pictures/placeHolderFHD.png}
    	\caption{Klávesnička}
   	\label{fig:keyborad}
\end{figure}

\subsection{Keyboard controller}
\lstinputlisting[language=C++, caption={Display controller}, label={lst:cpp_display}]{../code/src/components/Keyboard.h}

\subsection{Připojení k desce}
\begin{table} [h!]
	\centering
	\catcode`\-=12 % Because of czech
	\begin{tabular}[c]{|| c | c | c ||}
	\hline
		\multicolumn{3}{||c||}{Piny klávesničky} \\
	\hline
 		 \textbf{PIN} & \textbf{Tlačítko} & \textbf{Akce}\\
	\hline
		42 &  RIGHT & Posune menu doprava\\
	\hline
		46 & POWER & N/A\\
	\hline
		48 & LEFT & Posune menu doleva\\
	\hline
		50 & MENU & N/A\\
	\hline
		52 & AUTO & Začne testovat vybraný kabel\\
	\hline
	\end{tabular}
	\caption{Pinové rozložení klávesnice}
	\label{table:pinKB}
\end{table}